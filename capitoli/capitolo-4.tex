% !TEX encoding = UTF-8
% !TEX TS-program = pdflatex
% !TEX root = ../tesi.tex

%**************************************************************
\chapter{Valutazioni retrospettive}
\label{cap:valutazioni-retrospettive}
%**************************************************************

\section{Soddisfacimento degli obiettivi}

Elenco formale degli obiettivi soddisfatti e descrizione del grado di soddisfacimento di questi. In questo capitolo riassumerò anche i prodotti di stage completati, ossia applicazione, containerizzazione del \textit{back-end} e documentazione. \\
Il bilancio degli obiettivi verrà trattato sia dal punto di vista degli obiettivi pattuiti con l'azienda che dal punto di vista dei miei obiettivi personali.

%**************************************************************

\section{Bilancio formativo}

\subsection{Maturazione professionale}

Descrizione delle competenze professionali acquisite, sia in termini di tecnologie imparate che di abilità nella gestione del lavoro. In questo capitolo parlerò dunque dei linguaggi di programmazione, dei framework e delle infrastrutture che ho conosciuto e imparato a utilizzare e/o di cui ho imparato l'utilizzo in un contesto reale come è il tirocinio in azienda.

\subsection{Rapporto tra università e lavoro}

Trattazione delle differenze tra quanto appreso in ambito accademico e quanto utilizzato durante il tirocinio in azienda. \\
Nella stesura di questo capitolo utilizzerò un doppio approccio alla trattazione: da un lato parlerò delle tecnologie utilizzate che non vengono insegnate durante la laurea triennale; dall'altro parlerò della distanza che ho riscontrato tra le conoscenze "puramente accademiche" e l'applicazione di queste in contesto di lavoro reale.

%**************************************************************
