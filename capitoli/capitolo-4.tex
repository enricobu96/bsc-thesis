% !TEX encoding = UTF-8
% !TEX TS-program = pdflatex
% !TEX root = ../tesi.tex

%**************************************************************
\chapter{Valutazioni retrospettive}
\label{cap:valutazioni-retrospettive}
%**************************************************************

\section{Soddisfacimento degli obiettivi}

Una volta conclusi tutti i processi e le attività di sviluppo e documentazione ho analizzato insieme al tutor aziendale gli obiettivi fissati ad inizio \textit{stage}, al fine di individuare il grado di soddisfacimento di questi e poter effettuare un'analisi a posteriori di cosa avrei potuto gestire meglio. In questa tabella sono riportati gli obiettivi fissati, il loro grado di soddisfacimento e una breve nota che sintetizza questo soddisfacimento. \\

\begin{table}[h]
  \begin{center}
\begin{tabular}{lll}
\textbf{Obiettivo}         & \textbf{Risultato}               & \textbf{Note}            \\ \hline
\multicolumn{1}{|l|}{\texttt{O-O1}} & \multicolumn{1}{l|}{Soddisfatto} & \multicolumn{1}{l|}{ABC} \\ \hline
\multicolumn{1}{|l|}{\texttt{O-O2}} & \multicolumn{1}{l|}{Soddisfatto} & \multicolumn{1}{l|}{DEF} \\ \hline
\multicolumn{1}{|l|}{\texttt{O-O3}} & \multicolumn{1}{l|}{Soddisfatto} & \multicolumn{1}{l|}{GHI} \\ \hline
\multicolumn{1}{|l|}{\texttt{O-D1}} & \multicolumn{1}{l|}{Soddisfatto} & \multicolumn{1}{l|}{LMN} \\ \hline
\multicolumn{1}{|l|}{\texttt{O-D2}} & \multicolumn{1}{l|}{Soddisfatto} & \multicolumn{1}{l|}{OPQ} \\ \hline
\multicolumn{1}{|l|}{\texttt{O-F1}} & \multicolumn{1}{l|}{Soddisfatto} & \multicolumn{1}{l|}{RST} \\ \hline
\multicolumn{1}{|l|}{\texttt{O-F2}} & \multicolumn{1}{l|}{Soddisfatto} & \multicolumn{1}{l|}{UVZ} \\ \hline
\end{tabular}
\end{center}
\caption{Grado di soddisfacimento degli obiettivi di \textit{stage}.}
\end{table}

Mi sono inoltre impegnato a consegnare tutti i prodotti richiesti dall'azienda; questi sono riassunti dalla seguente tabella.

\begin{table}[h]
    \begin{center}
\begin{tabularx}{\textwidth}{lX}
\textbf{Prodotto}                        & \textbf{Note}                                                                                                                                                                                                                                                                                                                                    \\ \hline
\multicolumn{1}{|l|}{Codice}             & \multicolumn{1}{l|}{Durante tutto il processo di sviluppo ho utilizzato il repository aziendale riservato agli stage. Ho quindi consegnato ufficialmente il codice effettuando il merge dei branch da me creati sul master branch}                                                                                                               \\ \hline
\multicolumn{1}{|l|}{Documentazione}     & \multicolumn{1}{l|}{La documentazione che ho consegnato consiste nella documentazione tecnica. Questa è integrata nei file contenenti il codice, e ho provveduto ad automatizzarne la visualizzazione, con il tool Compodoc, tramite la creazione di uno script npm}                                                                             \\ \hline
\multicolumn{1}{|l|}{Containerizzazione} & \multicolumn{1}{l|}{Per quanto concerne la containerizzazione, ho rilasciato sul repository precedentemente citato due file yml che, eseguiti tramite docker compose, costruiscono e mettono a disposizione rispettivamente la componente back-end di SyncTrace e le interfacce grafiche per la gestione dei microservizi Spring e del database} \\ \hline
\end{tabularx}
\end{center}
\end{table}
% Alla conclusione dello stage ho analizzato insieme al mio tutor...
% Tabella obiettivi con descrizione
% Per quanto riguarda i prodotti completati...applicazione (pienamente funzionante), containerizzazione (due file yml), documentazione (doc coverage 100%)
% Posso quindi dirmi soddisfatto perché ho soddisfatto tutto. Unica pecca mancanza di alcune parti di documentazione + praticamente solo smartworking

%**************************************************************

\section{Bilancio formativo}

\subsection{Maturazione professionale}

% durante lo stage ho approfondito molte cose...
% per quanto riguarda le tecnologie...
% - approfondimento di typescript
% - angular e ionic
% - docker
% - in minor parte java, spring, postgres
% Inoltre...scrum
% Infine...gestione del lavoro & lavoro reale

% Descrizione delle competenze professionali acquisite, sia in termini di tecnologie imparate che di abilità nella gestione del lavoro. In questo capitolo parlerò dunque dei linguaggi di programmazione, dei framework e delle infrastrutture che ho conosciuto e imparato a utilizzare e/o di cui ho imparato l'utilizzo in un contesto reale come è il tirocinio in azienda.

\subsection{Rapporto tra università e lavoro}

% forma mentis e attitudine al lavoro, soprattutto grazie a SWE
% per quanto riguarda le tecnologie ho invece trovato una carenza, perché in università non ho mai avuto modo di studiare approfonditamente js e framework più all'avanguardia come angular, ad eccezione di ingegneria del software (che però non fa statistica perché non tutti fanno il progetto su ste robe e lo scopo principale del corso a mio parere non è apprendere nuove tecnologie ma sapersi organizzare). Buono html, carenza anche per css
% per quanto riguarda l'applicazione delle conoscenze accademiche al lavoro la cosa che mi è stata più utile è stata conoscere i processi di sviluppo blah blah

% Trattazione delle differenze tra quanto appreso in ambito accademico e quanto utilizzato durante il tirocinio in azienda. \\
% Nella stesura di questo capitolo utilizzerò un doppio approccio alla trattazione: da un lato parlerò delle tecnologie utilizzate che non vengono insegnate durante la laurea triennale; dall'altro parlerò della distanza che ho riscontrato tra le conoscenze "puramente accademiche" e l'applicazione di queste in contesto di lavoro reale.

%**************************************************************
