% !TEX encoding = UTF-8
% !TEX TS-program = pdflatex
% !TEX root = ../tesi.tex

%**************************************************************
\chapter{Il progetto nel contesto aziendale}
\label{cap:progetto-contesto-aziendale}
%**************************************************************

\section{Il rapporto tra stage e azienda}

Descrizione dell'approccio dell'azienda Sync Lab ai tirocini formativi (non unicamente al mio). \\
Ritengo importante parlare di questo poiché l'azienda in questione è particolarmente attenta e interessata agli \textit{stage} universitari come ottima fonte di possibili assunzioni e stimolo ad approfondire nuove tematiche e tecnologie.

%**************************************************************

\section{L'azienda in relazione al contesto attuale}

Descrizione del contesto attuale, ossia dell'emergenza sanitaria in corso. Ritengo sia utile, se non fondamentale, approfondire questa tematica per due motivi: il primo, già citato precedentemente, è il fatto che il mio tirocinio è coinciso temporalmente con l'emergenza sanitaria, e le modalità di lavoro ne sono state indubbiamente influenzate; il secondo, a mio parere più importante, è che il progetto che ho sviluppato è inerente all'emergenza sanitaria, poiché facente parte di un progetto più grande riguardante il \textit{contact tracing}.

%**************************************************************

\section{Lo scopo dello stage}

Descrizione più dettagliata dello scopo del mio particolare tirocinio; a partire dall'introduzione al contesto attuale del precedente capitolo, espliciterò in questa sezione lo scopo dell'applicativo oggetto del mio stage, calato nel progetto globale \textit{SyncTrace}.

%**************************************************************

\section{Vincoli e obiettivi dello stage}

Breve introduzione al capitolo, riportante gli accordi presi con il tutor aziendale e con il tutor interno e riportati nel piano di lavoro. In questa sezione introdurrò anche il fatto che il piano di lavoro è stato rimodulato in corso di progetto.

\subsection{Vincoli temporali}

Descrizione dei vincoli temporali, ossia quelli definiti nel piano di lavoro iniziale e la successiva modifica a questo. \\
Questa sezione conterrà la pianificazione dei periodi in cui il mio \textit{stage} è stato concettualmente diviso, ossia formazione, sviluppo e verifica, prendendo a riferimento il piano di lavoro e le sue successive modifiche.

\subsection{Vincoli organizzativi}

Descrizione dei vincoli organizzativi. In questo capitolo parlerò delle modalità organizzative che ho dovuto seguire, in particolare gli allineamenti settimanali con il tutor aziendale, il sistema di registro delle attività svolte e l'utilizzo degli strumenti di configurazione quali la piattaforma GitLab.

\subsection{Vincoli tecnologici}

Descrizione dei vincoli tecnologici del progetto. In questo capitolo parlerò delle tecnologie a cui lo sviluppo del mio progetto è stato vincolato.

\subsection{Obiettivi}

Elenco e descrizione degli obiettivi fissati. Questi saranno elencati in maniera formale, al fine di poter discutere correttamente del raggiungimento di ogni obiettivo nel capitolo \S 3, alla sezione riguardante gli obiettivi raggiunti.

%**************************************************************

\section{Motivazione della scelta}

Descrizione delle motivazioni personali che mi hanno portato alla scelta dell'azienda e dello \textit{stage} in particolare. \\
In questo capitolo partirò parlando dell'evento \textit{StageIT} e delle offerte di tirocinio che mi sono state presentate, arrivando poi al perché io abbia scelto Sync Lab e, in particolare, questo progetto.

%**************************************************************

\section{Formazione}

Breve introduzione al periodo di formazione all'interno dell'azienda, contenente un riassunto del percorso di formazione che ho svolto e le sue modalità.

\subsection{Tecnologie}

Analisi dettagliata sulle tecnologie studiate, con enfasi sull'applicazione di ogni tecnologia all'interno del mio progetto. A ogni tecnologia corrisponderà una sottosezione non numerata.

\subsection{Progetto}

Analisi di quanto svolto nel periodo antecedente al mio tirocinio per quanto concerne il progetto \textit{SyncTrace}, enfatizzando il fatto che il mio progetto di stage è consistito nel \textit{porting} di un applicativo web su dispositivi mobili (e quindi analizzando più approfonditamente proprio l'applicativo web).

%**************************************************************
