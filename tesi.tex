% !TEX encoding = UTF-8
% !TEX TS-program = pdflatex
% !TEX root = tesi.tex
% !TEX spellcheck = it-IT

% PDF/A filecontents
\RequirePackage{filecontents}
\begin{filecontents*}{\jobname.xmpdata}
  \Title{Document’s title}
  \Author{Author’s name}
  \Language{it-IT}
  \Subject{The abstract, or short description.}
  \Keywords{keyword1\sep keyword2\sep keyword3}
\end{filecontents*}

\documentclass[10pt,                    % corpo del font principale
               a4paper,                 % carta A4
               twoside,                 % impagina per fronte-retro
               openright,               % inizio capitoli a destra
               english,
               italian,
               ]{book}

%**************************************************************
% Importazione package
%**************************************************************

\PassOptionsToPackage{dvipsnames}{xcolor} % colori PDF/A

\usepackage{colorprofiles}
\usepackage{multirow}

\usepackage[a-2b,mathxmp]{pdfx}[2018/12/22]
                                        % configurazione PDF/A
                                        % validare in https://www.pdf-online.com/osa/validate.aspx

%\usepackage{amsmath,amssymb,amsthm}    % matematica

\usepackage[T1]{fontenc}                % codifica dei font:
                                        % NOTA BENE! richiede una distribuzione *completa* di LaTeX

\usepackage[utf8]{inputenc}             % codifica di input; anche [latin1] va bene
                                        % NOTA BENE! va accordata con le preferenze dell'editor

\usepackage[english, italian]{babel}    % per scrivere in italiano e in inglese;
                                        % l'ultima lingua (l'italiano) risulta predefinita

\usepackage{bookmark}                   % segnalibri

\usepackage{caption}                    % didascalie

\usepackage{chngpage,calc}              % centra il frontespizio

\usepackage{csquotes}                   % gestisce automaticamente i caratteri (")

\usepackage{emptypage}                  % pagine vuote senza testatina e piede di pagina

\usepackage{epigraph}			% per epigrafi

\usepackage{eurosym}                    % simbolo dell'euro

%\usepackage{indentfirst}               % rientra il primo paragrafo di ogni sezione

\usepackage{graphicx}                   % immagini

\usepackage{hyperref}                   % collegamenti ipertestuali

\usepackage[binding=5mm]{layaureo}      % margini ottimizzati per l'A4; rilegatura di 5 mm

\usepackage{listings}                   % codici

\usepackage{microtype}                  % microtipografia

\usepackage{mparhack,fixltx2e,relsize}  % finezze tipografiche

\usepackage{nameref}                    % visualizza nome dei riferimenti
\usepackage[font=small]{quoting}        % citazioni

\usepackage{subfig}                     % sottofigure, sottotabelle

\usepackage[italian]{varioref}          % riferimenti completi della pagina

\usepackage{booktabs}                   % tabelle
\usepackage{tabularx}                   % tabelle di larghezza prefissata
\usepackage{longtable}                  % tabelle su più pagine
\usepackage{ltxtable}                   % tabelle su più pagine e adattabili in larghezza

\usepackage[toc, acronym]{glossaries}   % glossario
                                        % per includerlo nel documento bisogna:
                                        % 1. compilare una prima volta tesi.tex;
                                        % 2. eseguire: makeindex -s tesi.ist -t tesi.glg -o tesi.gls tesi.glo
                                        % 3. eseguire: makeindex -s tesi.ist -t tesi.alg -o tesi.acr tesi.acn
                                        % 4. compilare due volte tesi.tex.

\usepackage[backend=biber,style=verbose-ibid,hyperref,backref]{biblatex}
                                        % eccellente pacchetto per la bibliografia;
                                        % produce uno stile di citazione autore-anno;
                                        % lo stile "numeric-comp" produce riferimenti numerici
                                        % per includerlo nel documento bisogna:
                                        % 1. compilare una prima volta tesi.tex;
                                        % 2. eseguire: biber tesi
                                        % 3. compilare ancora tesi.tex.


\renewcommand\tabularxcolumn[1]{m{#1}}% for vertical centering text in X column

% LISTING JavaScript %

\lstdefinelanguage{JavaScript}{
  morekeywords=[1]{break, continue, delete, else, for, function, if, in,
    new, return, this, typeof, var, void, while, with},
  % Literals, primitive types, and reference types.
  morekeywords=[2]{false, null, true, boolean, number, undefined,
    Array, Boolean, Date, Math, Number, String, Object},
  % Built-ins.
  morekeywords=[3]{eval, parseInt, parseFloat, escape, unescape},
  sensitive,
  morecomment=[s]{/*}{*/},
  morecomment=[l]//,
  morecomment=[s]{/**}{*/}, % JavaDoc style comments
  morestring=[b]',
  morestring=[b]"
}[keywords, comments, strings]

\lstdefinelanguage[ECMAScript2015]{JavaScript}[]{JavaScript}{
  morekeywords=[1]{await, async, case, catch, class, const, default, do,
    enum, export, extends, finally, from, implements, import, instanceof,
    let, static, super, switch, throw, try},
  morestring=[b]` % Interpolation strings.
}

\definecolor{mediumgray}{rgb}{0.3, 0.4, 0.4}
\definecolor{mediumblue}{rgb}{0.0, 0.0, 0.8}
\definecolor{forestgreen}{rgb}{0.13, 0.55, 0.13}
\definecolor{darkviolet}{rgb}{0.58, 0.0, 0.83}
\definecolor{royalblue}{rgb}{0.25, 0.41, 0.88}
\definecolor{crimson}{rgb}{0.86, 0.8, 0.24}

\lstdefinestyle{JSES6Base}{
  backgroundcolor=\color{white},
  basicstyle=\ttfamily,
  breakatwhitespace=false,
  breaklines=false,
  captionpos=b,
  columns=fullflexible,
  commentstyle=\color{mediumgray}\upshape,
  emph={},
  emphstyle=\color{crimson},
  extendedchars=true,  % requires inputenc
  fontadjust=true,
  frame=single,
  identifierstyle=\color{black},
  keepspaces=true,
  keywordstyle=\color{mediumblue},
  keywordstyle={[2]\color{darkviolet}},
  keywordstyle={[3]\color{royalblue}},
  numbers=left,
  numbersep=5pt,
  numberstyle=\tiny\color{black},
  rulecolor=\color{black},
  showlines=true,
  showspaces=false,
  showstringspaces=false,
  showtabs=false,
  stringstyle=\color{forestgreen},
  tabsize=2,
  title=\lstname,
  upquote=true  % requires textcomp
}

\lstdefinestyle{JavaScript}{
  language=JavaScript,
  style=JSES6Base
}
\lstdefinestyle{ES6}{
  language=[ECMAScript2015]JavaScript,
  style=JSES6Base
}

\newcommand\ProcessThreeDashes{\llap{\color{cyan}\mdseries-{-}-}}

\input{tesi-config}                     % file con le impostazioni personali

\begin{document}
%**************************************************************
% Materiale iniziale
%**************************************************************
\frontmatter
\input{inizio-fine/frontespizio}
\input{inizio-fine/colophon}
% % !TEX encoding = UTF-8
% !TEX TS-program = pdflatex
% !TEX root = ../tesi.tex

%**************************************************************
% Dedica
%**************************************************************
\cleardoublepage
\phantomsection
\thispagestyle{empty}
\pdfbookmark{Dedica}{Dedica}

\vspace*{3cm}

\begin{center}
% Qua dentro mettere la citazione. \\ \medskip
% --- Persona Citata
\end{center}

\medskip

\begin{center}
% Qua dentro mettere dedica
\end{center}

% !TEX encoding = UTF-8
% !TEX TS-program = pdflatex
% !TEX root = ../tesi.tex

%**************************************************************
% Sommario
%**************************************************************
\cleardoublepage
\phantomsection
\pdfbookmark{Sommario}{Sommario}
\begingroup
\let\clearpage\relax
\let\cleardoublepage\relax
\let\cleardoublepage\relax

\chapter*{Sommario}

Breve descrizione del documento, della durata del tirocinio, del laureando e dell'azienda presso cui ho svolto lo \textit{stage}. \\
Breve riassunto degli obiettivi del tirocinio svolto.

%\vfill
%
%\selectlanguage{english}
%\pdfbookmark{Abstract}{Abstract}
%\chapter*{Abstract}
%
%\selectlanguage{italian}

\endgroup

\vfill

% !TEX encoding = UTF-8
% !TEX TS-program = pdflatex
% !TEX root = ../tesi.tex

%**************************************************************
% Ringraziamenti
%**************************************************************
\cleardoublepage
\phantomsection
\pdfbookmark{Ringraziamenti}{ringraziamenti}

\begin{flushright}{ % altra citazione
	\slshape
	``Boschi ed acque, venti ed alberi, saggezza, forza e cortesia, che il favore della giungla ti accompagni.''} \\
	\medskip
    --- Rudyard Kipling
\end{flushright}


\bigskip

\begingroup
\let\clearpage\relax
\let\cleardoublepage\relax
\let\cleardoublepage\relax

\chapter*{Ringraziamenti}

% \noindent \textit{Ringraziamento numero uno.}\\

% \noindent \textit{Ringraziamento numero due.}\\

% \noindent \textit{Ringraziamento numero tre.}\\
\bigskip

\noindent\textit{\myLocation, \myTime}
\hfill \myName

\endgroup

\input{inizio-fine/indici}
\cleardoublepage

%**************************************************************
% Materiale principale
%**************************************************************
\mainmatter
% !TEX encoding = UTF-8
% !TEX TS-program = pdflatex
% !TEX root = ../tesi.tex

%**************************************************************
\chapter{Contesto aziendale}
\label{cap:contesto-aziendale}
%**************************************************************

% \noindent Esempio di utilizzo di un termine nel glossario \\
% \gls{api}. \\
%
% \noindent Esempio di citazione in linea \\
% \cite{site:agile-manifesto}. \\
%
% \noindent Esempio di citazione nel pie' di pagina \\
% citazione\footcite{womak:lean-thinking} \\

%**************************************************************
\section{Introduzione}

In questo documento descrivo la mia esperienza di \textit{stage} presso la sede di Padova di \textbf{Sync Lab}, un'azienda nata come \textit{software house} e che, negli anni, si è affermata nel campo dell'\textit{ICT} (\textit{Information and Comunication Technology}).\\
È di particolare rilievo, nella mia esperienza, il periodo storico in cui questo tirocinio è avvenuto: ho infatti effettuato lo stage tra il mese di settembre e il mese di ottobre 2020, ossia durante l'emergenza sanitaria globale dovuta alla malattia \textit{COVID-19}. La pandemia in atto ha influenzato radicalmente il lavoro di molte persone, me compreso: al fine di ridurre le possibilità di contagio, infatti, ho dovuto svolgere gran parte del tirocinio in regime di \textit{smart working}, potendo accedere alla struttura solo una volta alla settimana, e alle infrastrutture quasi esclusivamente in maniera remota.
Nonostante la situazione non favorevole, ho avuto comunque modo di interfacciarmi con i dipendenti dell'azienda; sono quindi riuscito anche a raccogliere le informazioni riguardanti l'azienda che, unite a delle ricerche online, ho riportato in questo capitolo.

\section{L'azienda}

Sync Lab nasce come \textit{software house} a Napoli nel 2002; negli anni, l'azienda si espande a gran velocità, aprendo sedi a Roma, Milano, Padova e Verona. Al giorno d'oggi, l'azienda conta 5 sedi, per un totale di oltre 250 dipendenti e più di 150 clienti diretti e finali. \\
Dall'apertura ad oggi, Sync Lab si è tramutata in un \textit{system integrator} grazie alla maturazione delle competenze tecniche e metodologiche in ambito software. Un tratto distintivo dell'azienda è la grande attenzione posta alla gestione delle \textbf{risorse umane}: testimonianza di ciò è il basso \textit{turn-over}, segno che i collaboratori condividono un progetto comune e concreto.
Altro segno d'eccellenza sono le certificazioni di qualità che l'azienda ha conseguito; finora, infatti, l'azienda ha ottenuto le certificazioni per gli standard \textit{ISO 9001}, \textit{ISO 14001}, \textit{ISO 27001} e \textit{ISO 45001}.

\subsection*{Prodotti e servizi offerti}

Una certa attenzione è posta dall'azienda alla diversificazione dei prodotti e dei servizi offerti; essi sono infatti inquadrabili in diverse aree tematiche, quali salute, \textit{privacy}, sicurezza, telecomunicazioni, finanza, territorio e ambiente. \\
Alcuni dei prodotti che l'azienda offre al momento sono i seguenti:

\begin{itemize}
  \item \textbf{DPS 4.0}: software per la gestione degli adempimenti alla \textit{Privacy GDPR - General Data Protection Regulation}, utilizzato da svariate aziende per attuare correttamente quanto previsto da tale regolamento europeo; tale prodotto permette di censire, tracciare e controllare chi può trattare dati personali in azienda;

  \item \textbf{StreamLog}: sistema finalizzato al soddisfacimento dei requisiti fissati dal \textit{Garante per la Protezione dei dati personali}, utilizzato dagli amministratori di sistema per controllare gli accessi agli utenti al fine di soddisfare i requisiti di \textit{privacy} richiesti dal garante;

  \item \textbf{SynClinic}: sistema informativo per la gestione integrata dei processi, clinici e amministrativi, di ospedali, cliniche e case di cura. Questo applicativo fornisce svariate funzionalità organizzate in moduli, elencati nell'immagine seguente, che intersecano i bisogni del personale amministrativo e quelli del personale clinico delle strutture sanitarie, permettendo di gestire e monitorare tutte le fasi del percorso di cura del paziente. È utilizzabile sia in \textit{cloud} che \textit{on premises};

  \begin{minipage}{\linewidth}
    \centering
      \includegraphics[height=6cm]{immagini/synclinic}
    \captionof{figure}{Moduli di SynClinic.}
    \caption*{\textbf{Fonte:} synclinic.it}
  \end{minipage}

  \item \textbf{StreamCrusher}: tecnologia che aiuta le aziende a effettuare corrette decisioni di \textit{business}, a identificare tempestivamente eventuali criticità e a riorganizzare i processi in base a nuove esigenze. Questo software è in grado di raccogliere, indicizzare e interpretare i dati, siano essi \textit{log} di applicazione o di sistema, \textit{alert}, dati di configurazione o modifiche ai sistemi, al fine di estrapolarne informazioni utili all'\textit{IT management};

  \item \textbf{Wave}: software che si propone come integrazione tra il mondo della videosorveglianza e quello dei \textit{Sistemi Informativi Territoriali}, permettendo di avere una visione geo-referenziata della distribuzione delle telecamere installate sul territorio e consentendo così all'utilizzatore di avere un impatto visivo immediato sull'area di copertura di una data installazione reale, o di avere un'anteprima di tale area in fase di progettazione;

  \item \textbf{Seastream}: piattaforma pensata per migliorare l'efficienza, la sicurezza e il processo di innovazione del settore marittimo; per fare ciò l'azienda fornisce, attraverso questa piattaforma, un \textit{Fleet Operation Center (FOC)}, ovvero un sistema di monitoraggio avanzato delle flotte armatoriali operative in tutto il mondo, e un \textit{Harbor Operation Platform (HOC)}, ovvero una piattaforma di servizi per gli operatori portuali. \\

\end{itemize}

\subsection*{Clienti principali}
Sync Lab collabora con numerose aziende italiane e multinazionali, sia pubbliche che private. Tra le aziende \textbf{private} più importanti possiamo trovare \textit{Sky}, \textit{Eni}, \textit{Enel}, \textit{Vodafone}, \textit{Accenture}, \textit{Fastweb}, \textit{Tim}, \textit{UniCredit} e \textit{H\&M}. \\
Tra le collaborazioni con \textbf{enti statali} e parastatali, invece, troviamo quelle con \textit{Trenitalia}, \textit{RAI}, \textit{Poste Italiane}, la \textit{Regione Lazio} e il \textit{Ministero dell'Economia e delle Finanze}.

%**************************************************************
\section{Processi aziendali}

\subsection{Processi}

L'azienda persegue i propri obiettivi attuando i processi di seguito elencati.

\subsubsection*{Consulenza}

L'azienda fornisce servizi di consulenza informatica a svariate imprese, sia pubbliche che private; questi servizi hanno lo scopo di far evolvere, sia in termini di sviluppo che di competitività, i clienti dell azienda. Per fare ciò, Sync Lab collabora con altre aziende di consulenza e con specialisti del settore.

\subsubsection*{Fornitura}

Il processo di fornitura viene istanziato ogniqualvolta un cliente assume Sync Lab per lo sviluppo e la realizzazione di un prodotto. Contemporaneamente alla realizzazione, l'azienda svolge delle attività che possano migliorare questo processo. In particolare:
\begin{itemize}
  \item \textbf{Qualità del software}: il software viene controllato e ottimizzato, anche attraverso l'utilizzo delle \textit{best practice}, al fine di aderire alle regole aziendali;
  \item \textbf{Verifica delle procedure}: le procedure vengono verificate, al fine di poter agire in maniera correttiva nel caso in cui si dovessero manifestare dei problemi;
  \item \textbf{Analisi e miglioramento degli standard}: gli standard aziendali vengono analizzati e, possibilmente, migliorati; da ciò ne consegue un costante miglioramento anche dal punto di vista di qualità del software.
\end{itemize}

\subsubsection*{Sviluppo}

Per quanto riguarda il processo di sviluppo, Sync Lab fa uso della metodologia \textit{Agile}\footcite{tec:agile}, descritta più dettagliatamente in seguito. Questa permette di coinvolgere il cliente durante tutto il processo, tenendolo aggiornato sull'evoluzione dello sviluppo del prodotto e ricevendo di ritorno le sue opinioni, al fine di poter agire sia correttivamente che migliorativamente.

\subsubsection*{Manutenzione}

Una volta consegnato il prodotto, l'azienda assicura le attività di manutenzione per tutto il ciclo di vita del software. La manutenzione offerta dall'azienda è di tre tipi:
\begin{itemize}
  \item \textbf{Correttiva}, corrispondente alla correzione di eventuali difetti;
  \item \textbf{Adattiva}, ossia il riadattamento del software a nuovi requisiti quali l'ambiente di produzione o l'architettura;
  \item \textbf{Evolutiva}, ossia l'aggiunta o l'aggiornamento in senso migliorativo di porzioni di software.
\end{itemize}

\subsection{Metodologia Agile}

Per il processo di sviluppo, l'azienda fa uso di una metodologia \textit{Agile} che molto si avvicina al modello \textit{Scrum}\footcite{tec:scrum}. Punto cardine del metodo di sviluppo di Sync Lab, infatti, è la continua interazione con gli \textit{stakeholder}, ossia i clienti: questi vengono coinvolti durante tutto il processo, venendo aggiornati sull'evoluzione del prodotto; questo permette all'azienda di ricevere \textit{feedback} che possono aiutare a migliorare il prodotto e ad adattarlo al meglio alle esigenze. \\
Come ho potuto constatare di persona durante il periodo di tirocinio, il modello adottato dall'azienda prevede uno sviluppo che procede per \textit{sprint}, ossia un'unità di base di durata fissa compresa tra una e quattro settimane a seconda degli obiettivi posti. A ogni \textit{sprint} corrisponde una funzionalità; questa viene verificata insieme al cliente, per tastarne la soddisfazione o ricevere consigli che possano migliorare tale nuova funzionalità. \\

\begin{minipage}{\linewidth}
  \centering
    \includegraphics[height=6cm]{immagini/scrumprocess}
  \captionof{figure}{Metodologia Scrum.}
  \caption*{\textbf{Fonte:} antevenio.com}
\end{minipage} \\

Lo sviluppo di un prodotto in Sync Lab passa attraverso cinque attività, inquadrabili tutte in quanto previsto dalla metodologia \textit{Scrum} riassunta dalla precedente immagine. \\
La prima attività che viene svolta è la redazione di una lista di cose da fare per portare a termine il progetto; nel caso del mio progetto di \textit{stage}, ad esempio, ho definito insieme al tutor aziendale le \textit{feature} che avrei dovuto mettere a disposizione, e abbiamo visionato assieme i \textit{bug} presenti nel codice che avrei dovuto utilizzare come \textit{baseline}, al fine di sapere cosa avrei dovuto correggere di quanto già fatto. Nella metodologia \textit{Scrum}, questa lista assume il nome di \textbf{Product Backlog}. \\
Dopo aver redatto questa lista, solitamente viene realizzata una pianificazione degli \textit{sprint} che è necessario effettuare per portare a termine il progetto. In ambito \textit{Scrum} questa pianificazione viene chiamata \textbf{Sprint Planning}, e nel caso del mio tirocinio è corrisposta alla divisione in incrementi che avrei dovuto effettuare per completare il piano di lavoro. Successivamente, vengono individuati dei sottoinsiemi di obiettivi per ogni \textit{sprint} definito durante l'attività chiamata \textbf{Sprint Backlog}. Un esempio pratico di questa attività è la definizione degli incrementi che ho effettuato prima di cominciare lo \textit{stage}. \\
Una volta definiti tutti gli obiettivi e aver correttamente diviso le tempistiche in \textbf{Sprint}, questi vengono ad uno ad uno eseguiti; nell'ambito del mio stage, questo è corrisposto allo sviluppo di quanto previsto dal piano di progetto. All'esecuzione di ogni incremento, inoltre, di questo è stata verificata l'aderenza agli obiettivi posti, che nella metodologia \textit{Scrum} vegnono chiamati \textbf{Sprint Goal}.

Durante il mio \textit{stage} ho inoltre potuto sperimentare anche due degli eventi facenti parte della metodologia \textit{Scrum}, ossia il \textit{Daily Scrum} e lo \textit{Sprint Review}:
\begin{itemize}
  \item Il \textbf{Daily Scrum}, come definito dai creatori di questa metodologia, è un evento a cadenza giornaliera in cui i membri del gruppo di lavoro discutono dell'andamento dello \textit{sprint}. Questo evento viene compiuto dai diversi \textit{team} di lavoro dell'azienda giornalmente; nel caso specifico del mio tirocinio, essendosi questo svolto in gran parte in \textit{remote working}, questo evento non ha potuto avere cadenza giornaliera. Ciononostante, ho effettuato svariati allineamenti con il tutor aziendale, e questo si può almeno in parte configurare con quanto previsto dalla metodologia;
  \item Lo \textbf{Sprint Review}, ossia la verifica di quanto effettuato durante uno \textit{sprint} alla fine di questo, viene effettuato da tutti i gruppi di lavoro al raggiungimento degli obiettivi fissati dallo \textit{sprint}. Nel caso specifico del mio tirocinio, questo evento ha avuto luogo quasi settimanalmente, con la verifica di quanto fatto durante l'incremento definito.
\end{itemize}

%**************************************************************

\section{Tecnologie utilizzate}

Come riportato sul sito aziendale, Sync Lab fa uso di diversi linguaggi di programmazione, \textit{framework} e strumenti di supporto moderni e funzionali, al fine di soddisfare i clienti; oltre a questo, l'azienda è costantemente aggiornata sulle tecnologie di riferimento e pronta ad espandere le proprie conoscenze con le tecnologie più moderne. \newpage

\begin{minipage}{\linewidth}
  \centering
    \includegraphics[height=4cm]{immagini/linguaggi}
  \captionof{figure}{Panoramica delle tecnologie utilizzate da Sync Lab.}
  \caption*{\textbf{Fonte:} synclab.it}
\end{minipage} \\

Tra i linguaggi di programmazione più utilizzati troviamo i seguenti:
\begin{itemize}
  \item \textbf{Java}\footcite{tec:java}: linguaggio di programmazione ad alto livello orientato agli oggetti, largamente utilizzato dalle aziende; in particolare, \textit{Java} viene utilizzato da Sync Lab, in accoppiata al \textit{framework Spring}\footcite{tec:spring}, per lo sviluppo dei servizi \textit{REST} necessari alle componenti \textit{back-end} di svariati applicativi, tra cui ad esempio \textit{SynClinic} e \textit{StreamCrusher}. Questo linguaggio, abbinato al \textit{framework Spring}, è stato utilizzato anche per lo sviluppo della componente \textit{back-end} dell'applicativo \textit{SyncTrace}, oggetto del mio tirocinio;

  \item \textbf{JavaScript}\footcite{tec:javascript}: linguaggio di programmazione orientato agli oggetti e agli eventi, originariamente pensato per la creazione di effetti dinamici interattivi per i siti web ma sempre più utilizzato come linguaggio \textit{general purpose} per lo sviluppo di applicativi web e non web. Viene utilizzato dall'azienda per realizzare la componente logica delle \textit{single page application};

  \item \textbf{TypeScript}\footcite{tec:typescript}: \textit{super-set} di \textit{JavaScript} sviluppato da \textit{Microsoft}. Questo linguaggio estende la sintassi di \textit{JavaScript} in modo che ogni programma scritto in \textit{JavaScript} possa funzionare anche con \textit{TypeScript}, e come il primo viene utilizzato dall'azienda per realizzare la componente logica delle \textit{single page application}. Si può trovare questo linguaggio, abbinato al \textit{framework Angular}\footcite{tec:angular}, in svariati prodotti dell'azienda, tra cui \textit{SynClinic} e \textit{DPS4.0}; è inoltre il linguaggio con il quale è stata scritta la componente \textit{front-end} della \textit{web application} di \textit{SyncTrace};

  \item \textbf{HTML5 e CSS3}\footcite{tec:htmlcss}: linguaggi di \textit{markup} utilizzati, anche dall'azienda, per modellare la componente visiva delle \textit{single page application} e, più in generale, dei siti web. Questi linguaggi di \textit{markup} sono usati dall'azienda principalmente nei progetti che coinvolgono il \textit{framework Angular}, come ad esempio \textit{SynClinic} e \textit{DPS4.0};

  \item \textbf{Kotlin}\footcite{tec:kotlin}: linguaggio di programmazione \textit{general purpose} e multi-paradigma, sviluppato da \textit{JetBrains}, utilizzato dall'azienda per sviluppare le applicazioni mobili per i dispositivi \textit{Android}. Un esempio di utilizzo di questo linguaggio è l'applicazione mobile per il \textit{contact tracing} del progetto \textit{SyncTrace};

  \item \textbf{Swift}\footcite{tec:swift}: linguaggio di programmazione orientato agli oggetti, sviluppato da \textit{Apple} e utilizzato dall'azienda per sviluppare le applicazioni mobili per i dispositivi \textit{iOS}.
\end{itemize}

L'azienda utilizza anche svariati \textit{framework} a supporto della programmazione; alcuni tra i \textit{framework} più utilizzati dall'azienda sono:
\begin{itemize}
  \item \textbf{Spring}: \textit{framework open-source} per lo sviluppo di applicazioni con linguaggio di programmazione \textit{Java}; viene utilizzato, possibilmente combinando il \textit{core} del \textit{framework} con altri progetti quali \textit{Spring Boot} e \textit{Spring Data}, per sviluppare applicativi lato \textit{server}. Esempio di utilizzo di questo \textit{framework} da parte di Sync Lab sono svariate applicazioni web la cui componente \textit{back-end} è sviluppata con l'ausilio di queste tecnologie, come ad esempio \textit{SynClinic} e \textit{StreamCrusher}. Di seguito viene riportata l'architettura generale di tale \textit{framework};

  \begin{minipage}{\linewidth}
    \centering
      \includegraphics[height=5cm]{immagini/spring}
    \captionof{figure}{Architettura dello \textit{Spring framework}.}
    \caption*{\textbf{Fonte:} javaboss.it}
  \end{minipage}

  \item \textbf{Angular}: \textit{framework open-source} per lo sviluppo di applicazioni web tramite linguaggio di programmazione \textit{TypeScript}. L'utilizzo principale di questa tecnologia risiede nello sviluppo di \textit{single page application} reattive e costruite su un \textit{back-end} composto da servizi \textit{REST}. Esempi di utilizzo di questo \textit{framework} sono \textit{SynClinic} e \textit{DPS4.0}, già citati in precedenza. \\
  Oltre alla possibilità di sviluppare applicativi veloci e funzionali, questo \textit{framework} offre anche un \textit{design pattern} di tipo \textit{Model-View-ViewModel} nativo come riassunto dalla seguente immagine, fattore che facilita la progettazione e lo sviluppo delle applicazioni;

  \begin{minipage}{\linewidth}
    \centering
      \includegraphics[height=5cm]{immagini/angular}
    \captionof{figure}{Pattern MVVM, adottato anche dal \textit{framework Angular}.}
    \caption*{\textbf{Fonte:} alphalogicinc.com}
  \end{minipage}

  \item \textbf{Electron}\footcite{tec:electron}: \textit{framework open-source} che consente lo sviluppo dell'interfaccia grafica di applicazioni desktop utilizzando tecnologie tipicamente pensate per il web, quali \textit{HTML}, \textit{CSS}, \textit{JavaScript} e \textit{TypeScript}; per fare ciò, questa tecnologia combina il motore di rendering \textit{Chromium} e il \textit{runtime} \textit{NodeJS}\footcite{tec:nodejs}. \\

  \begin{minipage}{\linewidth}
    \centering
      \includegraphics[height=2cm]{immagini/electron}
    \captionof{figure}{Funzionamento del \textit{framework Electron} con \textit{NodeJS}.}
    \caption*{\textbf{Fonte:} freecontent.manning.com}
  \end{minipage}
\end{itemize}

Sync Lab utilizza strumenti tecnologici anche per gestire il lavoro da remoto; in particolare, durante l'emergenza sanitaria ancora in corso, l'azienda fa uso di diverse tecnologie per organizzare il lavoro non in presenza, permettendo a tutti i collaboratori di rimanere correttamente aggiornati sulle attività e i compiti di cui sono responsabili. Alcune di queste tecnologie sono le seguenti:

\begin{itemize}
  \item \textbf{Google Meet}: servizio di \textit{Google} per effettuare videoconferenze online. Questa piattaforma viene usata dall'azienda, e in particolare è stata utilizzata anche durante il mio \textit{stage}, per comunicare con gli altri collaboratori e poter rimanere aggiornati sui progressi effettuati;

  \item \textbf{Discord}: applicazione \textit{VoIP} per la comunicazione vocale e testuale. Uno dei punti di forza di questa piattaforma, sfruttato anche dall'azienda, è la possibilità di poter avere più canali, sia testuali che vocali, all'interno dello stesso \textit{server}; questo permette una comunicazione più ordinata e metodica, riducendo il rischio di incomprensioni;

  \item \textbf{Google Docs}: servizio di \textit{Google} per la condivisione di documenti online. Questa piattaforma è stata utilizzata in particolare durante il mio tirocinio per tenere traccia degli incrementi giornalieri che ho svolto;

  \item \textbf{Trello}: software gestionale in stile \textit{Kanban}, utilizzato per coordinare il proprio \textit{workflow} e visualizzare quello degli altri collaboratori. Questa piattaforma si sposa bene con la metodologia \textit{Agile} che Sync Lab utilizza, in quanto può essere utilizzato come una \textit{Scrum board}.
\end{itemize}

%**************************************************************

\section{Propensione all'innovazione}

Sync Lab presta grande attenzione anche all'innovazione e allo sviluppo, sia in senso tecnologico che industriale. L'azienda, infatti, conta tre dipartimenti ideati per sperimentare e innovare, i quali sono \textbf{Research and Development}, nato con lo scopo di promuovere nuovi prodotti nati da ricerche in svariati settori, \textbf{Lab}, in cui l'azienda sviluppa soluzioni a quanto studiato nel precedente dipartimento, e \textbf{Start-up}, il cui scopo è quello di promuovere le \textit{start-up} di maggiore rilevanza per quanto riguarda l'innovazione; per fare ciò, Sync Lab collabora con svariati enti privati e università, sia italiane che estere. \\

\begin{minipage}{\linewidth}
  \centering
    \includegraphics[height=2cm]{immagini/universita}
  \captionof{figure}{Alcune università con le quali l'azienda collabora.}
  \caption*{\textbf{Fonte:} synclab.it}
\end{minipage} \\

Alcuni dei progetti di ricerca fondati e mantenuti da Sync Lab sono i seguenti:
\begin{itemize}
  \item \textbf{BIG-ASC}: acronimo di \textit{BIG Data and Advanced Analytics for Secure Mobile Commerce}, è un progetto che punta a creare una piattaforma \textit{Big Data} che sappia rispondere a requisiti stringenti delle piattaforme di \textit{Mobile Commerce}, come la scalabilità, l'autonomia e le \textit{performance}, attraverso l'analisi continua e in tempo reale dei dati d'utilizzo. Per fare questo, Sync Lab collabora con l'azienda \textit{CeRICT} e l'\textit{Università degli studi Parthenope} di Napoli;
  \item \textbf{eHealthNet}: ecosistema software per la Sanità Elettronica, che si propone di intervenire su quattro aree tematiche riguardanti la sanità: interoperabilità, pervasività, sostenibilità e preventivabilità. Per lo sviluppo di questo progetto è stato avviato un laboratorio che vede come collaboratori svariate aziende private, l'\textit{Istituto Italiano di Tecnologia}, l'\textit{Istituto Nazionale Tumori}, l'\textit{Università degli studi Federico II} di Napoli e l'\textit{Università degli studi di Salerno};
  \item \textbf{BDA4PHR}: piattaforma \textit{open-source}, scalabile, estendibile e manutenibile che offre servizi di \textit{storing} e \textit{Big Data Analytics} dedicati ad informazioni di tipo medico-sanitario. Lo scopo di questo progetto è la creazione di un \textit{repository} sicuro, distribuito e affidabile per la gestione, la condivisone e l'analisi di dati eterogenei. Tra i finanziatori di questo progetto ci sono l'\textit{Unione Europea} e il \textit{Ministero dello Sviluppo Economico} italiano. \\
\end{itemize}

I progetti a cui l'azienda prende parte sono molteplici e in continuo aumento; alcuni dei progetti ora attivi sono riassunti dal sito aziendale, del quale l'immagine che segue è una cattura di schermo.

\begin{minipage}{\linewidth}
  \centering
    \includegraphics[height=5.5cm]{immagini/progetti}
  \captionof{figure}{I progetti di ricerca e sviluppo a cui Sync Lab collabora.}
  \caption*{\textbf{Fonte:} synclab.it}
\end{minipage} \\

Sempre nell'ambito dell'innovazione, posso dire che l'azienda ha un atteggiamento molto propositivo e aperto alle nuove tecnologie: non è raro, infatti, che i collaboratori propongano l'utilizzo di nuovi linguaggi, \textit{framework} e strumenti per completare i servizi e i prodotti offerti dall'azienda. \\
Ho potuto respirare questo clima di apertura anche nell'ambito del mio tirocinio: avendo avuto accesso alla piattaforma \textit{Discord} aziendale, ho potuto notare che il \textit{server} è diviso in più sottocanali, ognuno dedicato a un ambito di sviluppo, in cui i dipendenti inoltrano articoli e documentazione riguardanti nuove tecnologie, aprendo così a un dibattito costruttivo.

% !TEX encoding = UTF-8
% !TEX TS-program = pdflatex
% !TEX root = ../tesi.tex

%**************************************************************
\chapter{Il progetto nel contesto aziendale}
\label{cap:progetto-contesto-aziendale}
%**************************************************************

\section{Il rapporto tra stage e azienda}

Descrizione dell'approccio dell'azienda Sync Lab ai tirocini formativi (non unicamente al mio). \\
Ritengo importante parlare di questo poiché l'azienda in questione è particolarmente attenta e interessata agli \textit{stage} universitari come ottima fonte di possibili assunzioni e stimolo ad approfondire nuove tematiche e tecnologie.

%**************************************************************

\section{L'azienda in relazione al contesto attuale}

Descrizione del contesto attuale, ossia dell'emergenza sanitaria in corso. Ritengo sia utile, se non fondamentale, approfondire questa tematica per due motivi: il primo, già citato precedentemente, è il fatto che il mio tirocinio è coinciso temporalmente con l'emergenza sanitaria, e le modalità di lavoro ne sono state indubbiamente influenzate; il secondo, a mio parere più importante, è che il progetto che ho sviluppato è inerente all'emergenza sanitaria, poiché facente parte di un progetto più grande riguardante il \textit{contact tracing}.

%**************************************************************

\section{Lo scopo dello stage}

Descrizione più dettagliata dello scopo del mio particolare tirocinio; a partire dall'introduzione al contesto attuale del precedente capitolo, espliciterò in questa sezione lo scopo dell'applicativo oggetto del mio stage, calato nel progetto globale \textit{SyncTrace}.

%**************************************************************

\section{Vincoli e obiettivi dello stage}

Breve introduzione al capitolo, riportante gli accordi presi con il tutor aziendale e con il tutor interno e riportati nel piano di lavoro. In questa sezione introdurrò anche il fatto che il piano di lavoro è stato rimodulato in corso di progetto.

\subsection{Vincoli temporali}

Descrizione dei vincoli temporali, ossia quelli definiti nel piano di lavoro iniziale e la successiva modifica a questo. \\
Questa sezione conterrà la pianificazione dei periodi in cui il mio \textit{stage} è stato concettualmente diviso, ossia formazione, sviluppo e verifica, prendendo a riferimento il piano di lavoro e le sue successive modifiche.

\subsection{Vincoli organizzativi}

Descrizione dei vincoli organizzativi. In questo capitolo parlerò delle modalità organizzative che ho dovuto seguire, in particolare gli allineamenti settimanali con il tutor aziendale, il sistema di registro delle attività svolte e l'utilizzo degli strumenti di configurazione quali la piattaforma GitLab.

\subsection{Vincoli tecnologici}

Descrizione dei vincoli tecnologici del progetto. In questo capitolo parlerò delle tecnologie a cui lo sviluppo del mio progetto è stato vincolato.

\subsection{Obiettivi}

Elenco e descrizione degli obiettivi fissati. Questi saranno elencati in maniera formale, al fine di poter discutere correttamente del raggiungimento di ogni obiettivo nel capitolo \S 3, alla sezione riguardante gli obiettivi raggiunti.

%**************************************************************

\section{Motivazione della scelta}

Descrizione delle motivazioni personali che mi hanno portato alla scelta dell'azienda e dello \textit{stage} in particolare. \\
In questo capitolo partirò parlando dell'evento \textit{StageIT} e delle offerte di tirocinio che mi sono state presentate, arrivando poi al perché io abbia scelto Sync Lab e, in particolare, questo progetto.

%**************************************************************

\section{Formazione}

Breve introduzione al periodo di formazione all'interno dell'azienda, contenente un riassunto del percorso di formazione che ho svolto e le sue modalità.

\subsection{Tecnologie}

Analisi dettagliata sulle tecnologie studiate, con enfasi sull'applicazione di ogni tecnologia all'interno del mio progetto. A ogni tecnologia corrisponderà una sottosezione non numerata.

\subsection{Progetto}

Analisi di quanto svolto nel periodo antecedente al mio tirocinio per quanto concerne il progetto \textit{SyncTrace}, enfatizzando il fatto che il mio progetto di stage è consistito nel \textit{porting} di un applicativo web su dispositivi mobili (e quindi analizzando più approfonditamente proprio l'applicativo web).

%**************************************************************

% !TEX encoding = UTF-8
% !TEX TS-program = pdflatex
% !TEX root = ../tesi.tex

%**************************************************************
\chapter{Il progetto di stage}
\label{cap:progetto-stage}
%**************************************************************

% !TEX encoding = UTF-8
% !TEX TS-program = pdflatex
% !TEX root = ../tesi.tex

%**************************************************************
\chapter{Valutazioni retrospettive}
\label{cap:valutazioni-retrospettive}
%**************************************************************

\section{Soddisfacimento degli obiettivi}

Una volta conclusi tutti i processi e le attività di sviluppo e documentazione ho analizzato insieme al tutor aziendale gli obiettivi fissati ad inizio \textit{stage}, al fine di individuare il grado di soddisfacimento di questi e poter effettuare un'analisi a posteriori di cosa avrei potuto gestire meglio. In questa tabella sono riportati gli obiettivi fissati, il loro grado di soddisfacimento e una breve nota che sintetizza questo soddisfacimento. \\

\begin{table}[h]
  \begin{center}
\begin{tabular}{lll}
\textbf{Obiettivo}         & \textbf{Risultato}               & \textbf{Note}            \\ \hline
\multicolumn{1}{|l|}{\texttt{O-O1}} & \multicolumn{1}{l|}{Soddisfatto} & \multicolumn{1}{l|}{ABC} \\ \hline
\multicolumn{1}{|l|}{\texttt{O-O2}} & \multicolumn{1}{l|}{Soddisfatto} & \multicolumn{1}{l|}{DEF} \\ \hline
\multicolumn{1}{|l|}{\texttt{O-O3}} & \multicolumn{1}{l|}{Soddisfatto} & \multicolumn{1}{l|}{GHI} \\ \hline
\multicolumn{1}{|l|}{\texttt{O-D1}} & \multicolumn{1}{l|}{Soddisfatto} & \multicolumn{1}{l|}{LMN} \\ \hline
\multicolumn{1}{|l|}{\texttt{O-D2}} & \multicolumn{1}{l|}{Soddisfatto} & \multicolumn{1}{l|}{OPQ} \\ \hline
\multicolumn{1}{|l|}{\texttt{O-F1}} & \multicolumn{1}{l|}{Soddisfatto} & \multicolumn{1}{l|}{RST} \\ \hline
\multicolumn{1}{|l|}{\texttt{O-F2}} & \multicolumn{1}{l|}{Soddisfatto} & \multicolumn{1}{l|}{UVZ} \\ \hline
\end{tabular}
\end{center}
\caption{Grado di soddisfacimento degli obiettivi di \textit{stage}.}
\end{table}

Mi sono inoltre impegnato a consegnare tutti i prodotti richiesti dall'azienda; questi sono riassunti dalla seguente tabella.

\begin{table}[h]
    \begin{center}
\begin{tabularx}{\textwidth}{lX}
\textbf{Prodotto}                        & \textbf{Note}                                                                                                                                                                                                                                                                                                                                    \\ \hline
\multicolumn{1}{|l|}{Codice}             & \multicolumn{1}{l|}{Durante tutto il processo di sviluppo ho utilizzato il repository aziendale riservato agli stage. Ho quindi consegnato ufficialmente il codice effettuando il merge dei branch da me creati sul master branch}                                                                                                               \\ \hline
\multicolumn{1}{|l|}{Documentazione}     & \multicolumn{1}{l|}{La documentazione che ho consegnato consiste nella documentazione tecnica. Questa è integrata nei file contenenti il codice, e ho provveduto ad automatizzarne la visualizzazione, con il tool Compodoc, tramite la creazione di uno script npm}                                                                             \\ \hline
\multicolumn{1}{|l|}{Containerizzazione} & \multicolumn{1}{l|}{Per quanto concerne la containerizzazione, ho rilasciato sul repository precedentemente citato due file yml che, eseguiti tramite docker compose, costruiscono e mettono a disposizione rispettivamente la componente back-end di SyncTrace e le interfacce grafiche per la gestione dei microservizi Spring e del database} \\ \hline
\end{tabularx}
\end{center}
\end{table}
% Alla conclusione dello stage ho analizzato insieme al mio tutor...
% Tabella obiettivi con descrizione
% Per quanto riguarda i prodotti completati...applicazione (pienamente funzionante), containerizzazione (due file yml), documentazione (doc coverage 100%)
% Posso quindi dirmi soddisfatto perché ho soddisfatto tutto. Unica pecca mancanza di alcune parti di documentazione + praticamente solo smartworking

%**************************************************************

\section{Bilancio formativo}

\subsection{Maturazione professionale}

% durante lo stage ho approfondito molte cose...
% per quanto riguarda le tecnologie...
% - approfondimento di typescript
% - angular e ionic
% - docker
% - in minor parte java, spring, postgres
% Inoltre...scrum
% Infine...gestione del lavoro & lavoro reale

% Descrizione delle competenze professionali acquisite, sia in termini di tecnologie imparate che di abilità nella gestione del lavoro. In questo capitolo parlerò dunque dei linguaggi di programmazione, dei framework e delle infrastrutture che ho conosciuto e imparato a utilizzare e/o di cui ho imparato l'utilizzo in un contesto reale come è il tirocinio in azienda.

\subsection{Rapporto tra università e lavoro}

% forma mentis e attitudine al lavoro, soprattutto grazie a SWE
% per quanto riguarda le tecnologie ho invece trovato una carenza, perché in università non ho mai avuto modo di studiare approfonditamente js e framework più all'avanguardia come angular, ad eccezione di ingegneria del software (che però non fa statistica perché non tutti fanno il progetto su ste robe e lo scopo principale del corso a mio parere non è apprendere nuove tecnologie ma sapersi organizzare). Buono html, carenza anche per css
% per quanto riguarda l'applicazione delle conoscenze accademiche al lavoro la cosa che mi è stata più utile è stata conoscere i processi di sviluppo blah blah

% Trattazione delle differenze tra quanto appreso in ambito accademico e quanto utilizzato durante il tirocinio in azienda. \\
% Nella stesura di questo capitolo utilizzerò un doppio approccio alla trattazione: da un lato parlerò delle tecnologie utilizzate che non vengono insegnate durante la laurea triennale; dall'altro parlerò della distanza che ho riscontrato tra le conoscenze "puramente accademiche" e l'applicazione di queste in contesto di lavoro reale.

%**************************************************************

% \appendix
% \input{capitoli/capitolo-A}             % Appendice A

%**************************************************************
% Materiale finale
%**************************************************************
% \backmatter
% \printglossaries
% \input{inizio-fine/bibliografia}
\end{document}
